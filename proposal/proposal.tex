\documentclass{scrartcl}

\usepackage[hidelinks]{hyperref}
\usepackage[none]{hyphenat}

\title{Essay Proposal}
\subtitle{COMP160 - Software Engineering Essay}

\author{1603294}

\begin{document}

\maketitle

\section*{Topic}

My essay will be on: Engineering Software for Accessibility

% Add details as appropriate.

\section*{Paper 1}
% This is an example! Replace the details with a paper relevant to your chosen topic.
\begin{description}
\item[Title:] A Mathematical Theory of Communication
\item[Citation:] \cite{shannon}
\item[Abstract:] ``The recent development of various methods of modulation such as PCM and PPM which exchange bandwidth for signal-to-noise ratio has intensified the interest in a general theory of communication. A basis for such a theory is contained in the important papers of Nyquist and Hartley on this subject. In the present paper we will extend the theory to include a number of new factors, in particular the effect of noise in the channel, and the savings possible due to the statistical structure of the original message and due to the nature of the final destination of the information.''
\item[Web link:] \url{http://ieeexplore.ieee.org/xpl/articleDetails.jsp?arnumber=6773024}
\item[Full text link:] \url{http://ieeexplore.ieee.org/stamp/stamp.jsp?tp=&arnumber=6773024}
\item[Comments:] This is a seminal article, which laid the foundations for what we now call information theory.
	It has been cited more than $76\,000$ times, making it one of the most-cited articles in computer science.
\end{description}

\section*{Paper 2}
\begin{description}
\item[Title:] Using benchmarking to advance research: a challenge to software engineering
\item[Citation:] \cite{bibtex_key}
\item[Abstract:] Benchmarks have been used in computer science to compare the performance of computer systems, information retrieval algorithms, databases, and many other technologies. The creation and widespread use of a benchmark within a research area is frequently accompanied by rapid technical progress and community building. These observations have led us to formulate a theory of benchmarking within scientific disciplines. Based on this theory, we challenge software engineering research to become more scientific and cohesive by working as a community to define benchmarks. In support of this challenge, we present a case study of the reverse engineering community, where we have successfully used benchmarks to advance the state of research.
\item[Web link:] http://ieeexplore.ieee.org/abstract/document/1201189/
\item[Full text link:] 
\item[Comments:] I found this article while seaching for software engineering in google scholar and as the article talks about testing the boundries of software engineering I decided that it would be an important aspect for improving accessibility.
\end{description}

\section*{Paper 3}
\begin{description}
\item[Title:] Title of paper
\item[Citation:] \cite{bibtex_key}
\item[Abstract:] Copy and paste the abstract here
\item[Web link:] Give the URL of the paper in IEEE Xplore, ACM Digital Library, or similar
\item[Full text link:] Give the URL of a downloadable PDF of the paper, if you can find one
\item[Comments:] Write a few sentences on how you found the article and why you believe it is relevant and/or important.
\end{description}

\section*{Paper 4}
\begin{description}
\item[Title:] Title of paper
\item[Citation:] \cite{bibtex_key}
\item[Abstract:] Copy and paste the abstract here
\item[Web link:] Give the URL of the paper in IEEE Xplore, ACM Digital Library, or similar
\item[Full text link:] Give the URL of a downloadable PDF of the paper, if you can find one
\item[Comments:] Write a few sentences on how you found the article and why you believe it is relevant and/or important.
\end{description}

\section*{Paper 5}
\begin{description}
\item[Title:] Title of paper
\item[Citation:] \cite{bibtex_key}
\item[Abstract:] Copy and paste the abstract here
\item[Web link:] Give the URL of the paper in IEEE Xplore, ACM Digital Library, or similar
\item[Full text link:] Give the URL of a downloadable PDF of the paper, if you can find one
\item[Comments:] Write a few sentences on how you found the article and why you believe it is relevant and/or important.
\end{description}

\bibliographystyle{ieeetran}
\bibliography{initial_references}

\end{document}
